\documentclass[a4paper,11pt]{report}
\usepackage{amsmath,amsthm,amssymb}
\usepackage[T1,T2A]{fontenc}
\usepackage[utf8]{inputenc}
\usepackage[english,russian]{babel}

\title{Двунаправленная очередь}
\author{Коротченок Остап Андреевич}

\begin{document}
\maketitle
\tableofcontents
\chapter{Структура}
Элементом двунаправленной очереди будет структура узел, в котором хранится переменная data, и переменная ссылка на следующий узел.

\chapter{Push/pop}
Н.У.О push\_front добавит узел в начало, сделает предыдущий first своим next, а сам станет новым first.
Н. У. О. pop\_front удалит первый элемент и вернет data удаленного. Node, записанный как next у удаленного, становится новым first.
\chapter{Сложность}
Добавление и удаление элементов константное, но их можно добавить(удалить) только в начало или конец. Обращение по индексам отсутствует. Поиск линейный, так как надо с первого элемента идти по next-ам и проверять.
\end{document}